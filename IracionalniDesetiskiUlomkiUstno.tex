\documentclass{beamer}

\usepackage[slovene]{babel}
\usepackage{amsfonts,amssymb}
\usepackage[utf8]{inputenc}
\usepackage{lmodern}
\usepackage[T1]{fontenc}

\usetheme{Warsaw}

\def\N{\mathbb{N}} % mnozica naravnih stevil
\def\Z{\mathbb{Z}} % mnozica celih stevil
\def\Q{\mathbb{Q}} % mnozica racionalnih stevil
\def\R{\mathbb{R}} % mnozica realnih stevil
\def\C{\mathbb{C}} % mnozica kompleksnih stevil


\def\qed{$\hfill\Box$}   % konec dokaza
\newtheorem{izrek}{Izrek}
\newtheorem{trditev}{Trditev}
\newtheorem{posledica}{Posledica}
\newtheorem{lema}{Lema}
\newtheorem{definicija}{Definicija}
\newtheorem{pripomba}{Pripomba}
\newtheorem{primer}{Primer}
\newtheorem{zgled}{Zgled}
\newtheorem{zgledi}{Zgledi uporabe}
\newtheorem{zglediaf}{Zgledi aritmetičnih funkcij}
\newtheorem{oznaka}{Oznaka}

\title{O nekaterih iracionalnih desetiških ulomkih}
\author{Gaja Jamnik}
\institute{Fakulteta za matematiko in fiziko \\
Oddelek za matematiko}
\date{2.\ april 2021}

\begin{document}
    
%%%%%%%%%%%%%%%%%%%%%%%%%%%%%%%%%%%%%%%%%%%%%%%%%%%%%%%%%%%%%%%%%%%%%%%%%%%%%%%%%%%

\begin{frame}
    \titlepage
\end{frame}

%%%%%%%%%%%%%%%%%%%%%%%%%%%%%%%%%%%%%%%%%%%%%%%%%%%%%%%%%%%%%%%%%%%%%%%%%%%%%%%%%%%

% Kot že naslov predstavitve pove, bomo danes govorili o desetiških ulomkih.
% Pa se za začetek spomnimo definicije.

\begin{frame}
    \frametitle{Uvod}
    \begin{definicija}
        \alert{Desetiški ulomek}
        je ulomek, katerega imenovalec je potenca števila $10$.
        \[ \frac{73}{1000}\]
        Za poenostavitev uporabljamo zapis z decimalno vejico:
        \[ 0,073\]
    \end{definicija}

    \pause

    Kakšna je razlika v decimalnem zapisu racionalnega in iracionalnega števila?

% Zdaj pa me zanima ali bi kdo povedal razliko v zapisu racionalnega in irracionalnega števila?
% To smo se naučili že v srednji, morda celo osnovni šoli.

    \pause
    \begin{itemize}
        \item racionalna števila: $5,6$;  $0,\overline{26}$
        \item irracionalna števila: $\pi = 3,1415926535 \dots$
    \end{itemize}
\end{frame}

% No iz vsakega zapisa desetiškega ulomka z neskončno decimalkami pa ne moremo zagotovo razbrati
% ali se po nekem členu decimalke periodično ponavljajo ali ne.
% Za pomoč pri obravnavi decimalnih števil, si poglejmo naslednjo definicijo.

\begin{frame}
    \frametitle{Uvod}
    \begin{definicija}
        Naj bo $x$ realno število, $ 0 < x < 1$, podano z decimalnim zapisom: 
        \[ 
            \begin{split}
                x & = \sum^n_{i=1} c_i 10^{-i} = \\
                & = 0,c_0c_1c_2 \cdots c_n ,
            \end{split}   
        \]
        kjer so $0 \leq c_i \leq 9 \ \forall i = 1, \dots, s$.
        
        Z $b$ označimo celo število sestavljeno iz zaporedja števk
        $b_1b_2b_3 \dots b_s$, 
        kjer je $s\geq 1$ in $0 \leq b_i \leq 9 \ \forall i = 1, \dots, s$.

        Pravimo, da število $x$ \alert{vsebuje blok števil} $(b) = (b_1b_2b_3 \dots)$, če obstaja $j \geq 0$, da je 
        $c_{i+j} = b_i$ za vse $i=1, 2, \dots s$. 
    \end{definicija}

    % ta definicija je popolni formalizem. Čisto intuitivno si bomo blok predstavljali
    % kot zaporedje števk, ki je vsebovan (ali ne) v decimalnem zapisu.

    \pause
    \begin{primer}
        $0,5934$ vsebuje blok $(593)$, ne vsebuje pa $(594)$ ali $(43)$.
    \end{primer} 


\end{frame}

\begin{frame}
    \frametitle{O številu $0,23571113\dots$}
    Poljubno decimalno število lahko razumemo kot zaporedje blokov celih števil.

    %Poglejmo si konkreten primer, ki nam bo služil kot motivacija za naprej
    \pause
    \[0,235711131719\dots\]
    % To število kot lahko prepoznate je decimalno število, ki je sestavljeno
    % iz zaporedja blokov praštevil.
    \pause
    %njegov zapis lahko ponazorimo kot:
    \[0,(p_1)(p_2)(p_3)(p_4)\dots\]
    $\{p_i\}$ predstavlja zaporedje praštevil.
    \pause
    \newline
    \newline
    Ali je to število iracionalno?

\end{frame}

\begin{frame}
    \frametitle{O številu $0,23571113\dots$}
    %Trdimo naslednje:
    \begin{trditev}
        Število $0,23571113 \dots$ je iracionalno.
    \end{trditev}
    %Za dokaz te trditve bomo potrebovali različico Dirichletovega izreka, ki pravi:
    \pause
    \begin{izrek}[Dirichletov izrek]
        \label{Dirichletov izrek}
        V vsakem zaporedju $ \lbrace an + b \rbrace_{n \in \N_0}$ naravnih števil, kjer sta $a$ in $b$
        tuji si naravni števili, je neskončno praštevil.
    \end{izrek}
    \pause
    %Najprej prilagodimo Dirichletov izrek za našo potrebo
    \emph{Dokaz:} Naj bo $s \geq$ poljubno celo število
    
    Po Dirichletovem izreku $ \{10^{s+1}k + 1\}$, $k \in \N$ vsebuje neskončno praštevil.
    %V Dirichletovem izreku vzemimo za a=10^{s+1} in za b=1
    % Ti števili sta si res tuji, zato bo to vredu
    \pause
    Obstajajo praštevila oblike:
    \[(k)\underbrace{00 \dots}_{s}1\]
    \pause
    Taka števila obstajajo za $\forall s \geq 0$.
    %Torej če večamo število s se bo večalo tudi število ničel, ki mu sledi ena enica
    %Ponazorimo s primerom
    % k_11 k_201 k_3001 k_40001... kjer so k_i neka naravna števila -> vedno več bo ničel
    $0,2357\dots$ je sestavljeno iz blokov takšne oblike, zato ne bo periodično.
    \qed

\end{frame}

\begin{frame}
    \frametitle{Decimalna števila z naraščajočimi bloki}
    % Do zdaj smo dokazali, da je decimalno število, ki ima za bloke zaporedje praštevil iracionalno
    % Zdaj pa nas zanima kaj lahko povemo o decimalnih številih, ki so sestavljena iz blokov
    % za neko drugo naraščajoče zaporedje
    \pause
    Naj bo $1 \leq a_1 < a_2 < \dots $ strogo naraščajoče zaporedje celih števil. 
    Označimo: \[Dec\{a_k\} = 0,(a_1)(a_2)(a_3)... \  \ ;  a_k \in \Z \ k \in \N \]
    \pause
    % Za začetek povejmo izrek, ki nam bo podal pogoj pod katerim bo tako decimalno število iracionalno

    \begin{izrek}
        Če za strogo naraščajoče zaporedje celih števil $\{a_i\}_{i \in \N}$ velja 
        \[ \sum_{i=1}^{\infty} \frac{1}{a_i} = \infty ,\]
        potem je $Dec\{a_k\}$ iracionalno.
    \end{izrek}

\end{frame}

\begin{frame}
    Ali število $0,23571113\dots$ zadošča pogoju izreka?
    \pause
    % V bistvu nas zanima ali bo veljalo, da naslednja vrsta konvergira
    \[
        \sum_{p \ \text{praštevilo}}\frac{1}{p} =
        \frac{1}{2} + \frac{1}{3} + \frac{1}{5} + \frac{1}{7} + \dots \overset{?}{=} \infty \]
    \pause
    Divergenco vrste $\sum_{p \ \text{praštevilo}}\frac{1}{p}$ je leta 1737 dokazal Leonhard Euler.

\end{frame}




\end{document}
