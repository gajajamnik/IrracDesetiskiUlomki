\documentclass{beamer}

\usepackage[slovene]{babel}
\usepackage{amsfonts,amssymb}
\usepackage[utf8]{inputenc}
\usepackage{lmodern}
\usepackage[T1]{fontenc}

\usepackage{icomma}

\usetheme{Warsaw}

\def\N{\mathbb{N}} % mnozica naravnih stevil
\def\Z{\mathbb{Z}} % mnozica celih stevil
\def\Q{\mathbb{Q}} % mnozica racionalnih stevil
\def\R{\mathbb{R}} % mnozica realnih stevil
\def\C{\mathbb{C}} % mnozica kompleksnih stevil


\def\qed{$\hfill\Box$}   % konec dokaza
\newtheorem{izrek}{Izrek}
\newtheorem{trditev}{Trditev}
\newtheorem{posledica}{Posledica}
\newtheorem{lema}{Lema}
\newtheorem{definicija}{Definicija}
\newtheorem{pripomba}{Pripomba}
\newtheorem{primer}{Primer}
\newtheorem{zgled}{Zgled}
\newtheorem{zgledi}{Zgledi uporabe}
\newtheorem{zglediaf}{Zgledi aritmetičnih funkcij}
\newtheorem{oznaka}{Oznaka}

\title{O nekaterih iracionalnih desetiških ulomkih}
\author{Gaja Jamnik}
\institute{Fakulteta za matematiko in fiziko \\
Oddelek za matematiko}
\date{2.\ april 2021}

\begin{document}
    
%%%%%%%%%%%%%%%%%%%%%%%%%%%%%%%%%%%%%%%%%%%%%%%%%%%%%%%%%%%%%%%%%%%%%%%%%%%%%%%%%%%

\begin{frame}
    \titlepage
\end{frame}

%%%%%%%%%%%%%%%%%%%%%%%%%%%%%%%% UVOD %%%%%%%%%%%%%%%%%%%%%%%%%%%%%%%%%%%%%%%%%%%

% Kot že naslov predstavitve pove, bomo danes govorili o desetiških ulomkih.
% Pa se za začetek spomnimo definicije.

\begin{frame}
    \frametitle{Uvod}
    \begin{definicija}
        \alert{Desetiški ulomek}
        je ulomek, katerega imenovalec je potenca števila $10$.

    \end{definicija}
    \[ \frac{73}{1000}\]
    Za poenostavitev uporabljamo zapis z decimalno vejico (\emph{decimalni zapis}):
    \[ 0,073\]
    \pause
    Kakšna je razlika v decimalnem zapisu racionalnega in iracionalnega števila?

% Zdaj pa me zanima ali bi kdo povedal razliko v zapisu racionalnega in irracionalnega števila?
% To smo se naučili že v srednji, morda celo osnovni šoli.

    \pause
    \begin{itemize}
        \item racionalna števila: $5,6$;  $0,\overline{26}$
        \item iracionalna števila: $\pi = 3,1415926535 \ldots$
    \end{itemize}
\end{frame}

% No iz vsakega zapisa desetiškega ulomka z neskončno decimalkami pa ne moremo zagotovo razbrati
% ali se po nekem členu decimalke periodično ponavljajo ali ne.
% Naš glavni cilj tega predavanja bo ugtovoiti kako iz decimalnega zapisa razberemo ali je
% število iracionalno.

% Za pomoč pri obravnavi decimalnih števil, si poglejmo naslednjo definicijo.

\begin{frame}
    \frametitle{Uvod}
    \begin{definicija}
        Naj bo $x$ realno število, $ 0 < x < 1$, podano z decimalnim zapisom: 
        \[ 
            \begin{split}
                x & = \sum^{\infty}_{i=1} c_i 10^{-i} = \\
                & = 0,c_1c_2c_3 \ldots ,
            \end{split}   
        \]
        kjer so $0 \leq c_i \leq 9$ za vse $ i = 1, 2, \ldots$.
        
        Z $b$ označimo naravno število sestavljeno iz zaporedja števk
        $b_1b_2b_3 \ldots b_s$, 
        kjer je $s\geq 1$ in $0 \leq b_i \leq 9$ za vse $i = 1, \ldots, s$.

        Pravimo, da število $x$ \alert{vsebuje blok števil} $(b) = (b_1b_2b_3 \dots b_s)$, če obstaja $j \geq 0$, da je 
        $c_{i+j} = b_i$ za vse $i=1, 2, \ldots s$. 
    \end{definicija}

    % Ta definicija je popolni formalizem. 
    % Čisto intuitivno si bomo blok predstavljali
    % kot zaporedje števk, ki je vsebovan (ali ne) v decimalnem zapisu.

    \pause
    \begin{primer}
        $0,5934$ vsebuje blok $(593)$, ne vsebuje pa $(594)$ ali $(43)$.
    \end{primer} 


\end{frame}

%%%%%%%%%%%%%%%%%%%%%%% O ŠTEVILU 0,23571113... %%%%%%%%%%%%%%%%%%%%%%%%%%%%%%%%
\begin{frame}
    \frametitle{O številu $0,23571113\ldots$}
    Poljubno decimalno število lahko razumemo kot zaporedje blokov naravnih števil.

    %Poglejmo si konkreten primer, ki nam bo služil kot motivacija za naprej
    \pause
    \[0,235711131719\ldots\]
    % Zdaj me zanima ali prepozna kdo zaporedje blokov iz katerega je sestavljeno to decimalno št.?
    % To število kot lahko prepoznate je sestavljeno iz zaporedja blokov praštevil.
    \pause
    % Njegov zapis lahko ponazorimo kot:
    \[0,(p_1)(p_2)(p_3)(p_4)\ldots\]
    $\{p_i\}$ predstavlja zaporedje praštevil.
    \pause
    \newline
    \newline
    % Za začetek se vprašajmo ali je to število iracionalno?
    Ali je to število iracionalno?

    %Trdili bomo naslednje:

\end{frame}

\begin{frame}
    \frametitle{O številu $0,23571113\ldots$}
    \begin{trditev}
        Število $0,23571113 \ldots$ je iracionalno.
    \end{trditev}

    %Za dokaz te trditve bomo potrebovali različico Dirichletovega izreka, ki pravi:
    \pause
    \begin{izrek}[Dirichletov izrek]
        \label{Dirichletov izrek}
        V vsakem zaporedju $ \lbrace an + b \rbrace_{n \in \N_0}$ naravnih števil, kjer sta $a$ in $b$
        tuji si naravni števili, je neskončno praštevil.
    \end{izrek}
    \pause
    %Najprej prilagodimo Dirichletov izrek za našo potrebo
    \emph{Dokaz:} Naj bo $s \geq 0$ poljubno celo število
    
    Po Dirichletovem izreku $ \{10^{s+1}k + 1\}_{k \in \N}$,  vsebuje neskončno praštevil.
    %V Dirichletovem izreku vzemimo za a=10^{s+1} in za b=1
    % Ti števili sta si res tuji, zato bo to vredu
    \pause
    Obstajajo praštevila oblike:
    \[(k)\underbrace{00 \ldots}_{s}1\]
    \pause
    Taka števila obstajajo za vsak $s \geq 0$.
    %Torej če večamo število s se bo večalo tudi število ničel, ki mu sledi ena enica
    %Ponazorimo s primerom
    % k_11 k_201 k_3001 k_40001... kjer so k_i neka naravna števila -> vedno več bo ničel
    
    $0,2357\ldots$ vsebuje vse bloke takšne oblike, zato ne bo periodično.
    \qed

\end{frame}

%%%%%%%%%%%%%%%%%%% DECIMALNA ŠTEVILA Z NARAŠČAJOČIMI BLOKI %%%%%%%%%%%%%%%%%%%%%%%%

\begin{frame}
    \frametitle{Decimalna števila z naraščajočimi bloki}

    % Do zdaj smo dokazali, da je decimalno število, ki ima za bloke zaporedje praštevil iracionalno
    % Zdaj pa nas zanima kaj lahko povemo o decimalnih številih, ki so sestavljena iz blokov
    % za neko drugo naraščajoče zaporedje
    Naj bo $1 \leq a_1 < a_2 < \dots $ strogo naraščajoče zaporedje naravnih števil. 
    Označimo: \[Dec\{a_k\} = 0,(a_1)(a_2)(a_3)\ldots \  \ ; k \in \N, \ a_k \in \N \]
    \pause

    % Za začetek povejmo izrek, ki nam bo podal pogoj pod katerim bo tako decimalno število iracionalno

    \begin{izrek}\label{izrek1clanek1}
        Če za strogo naraščajoče zaporedje naravnih števil $\{a_i\}_{i \in \N}$ velja 
        \[ \sum_{i=1}^{\infty} \frac{1}{a_i} = \infty ,\]
        potem je $Dec\{a_k\}$ iracionalno.
    \end{izrek}

\end{frame}

\begin{frame}\frametitle{Decimalna števila z naraščajočimi bloki}

    Ali število $0,23571113\ldots$ zadošča pogoju izreka?
    \pause
    
    % Torej nas zanima ali bo veljalo, da naslednja vrsta konvergira
    \[
        \sum_{p \ \text{praštevilo}}\frac{1}{p} =
        \frac{1}{2} + \frac{1}{3} + \frac{1}{5} + \frac{1}{7} + \ldots \overset{?}{=} \infty \]
    \pause
    %Tega danes ne bomo dokazali je pa divergenco te vrste dokazal Euler leta 1737

    Divergenco vrste $\sum_{p \ \text{praštevilo}}\frac{1}{p}$ je leta 1737 dokazal Leonhard Euler.

\end{frame}

\begin{frame}\frametitle{Decimalna števila z naraščajočimi bloki}

    % Za dokaz izreka bomo potrebovali naslednjo lemo
    \begin{lema}
        Naj bo $(b) = (b_1b_2b_3 \dots b_s)$ blok števil. Z $X = X(b_1b_2b_3 \dots b_s)$ označimo
        zaporedje vseh naravnih števil, ki ne vsebujejo bloka števil (b). Potem 
        $\sum_{x \in X}^{\infty} \frac{1}{x}$ konvergira.
    \end{lema}
    \emph{Dokaz leme:}
    \newline
    \newline
    \newline
    \newline
    \newline
    \newline
    \newline
    \newline
    \newline
    \newline
    \newline
    \newline
    \newline
\end{frame}

\begin{frame}
    % tu bo dokaz leme
\end{frame}

\begin{frame}
    \emph{Dokaz izreka:}
    
    $\{a_i\}_{i \in \N}$ strogo naraščajoče zaporedje naravnih števil
    
    dokazujemo: $\sum_{i=1}^{\infty} \frac{1}{a_i} = \infty \Rightarrow Dec\{a_k\} \in \R \setminus\Q$
    \newline
    \newline
    \newline
    \newline
    \newline
    \newline
    \newline
    \newline
    \newline
    \newline
    \newline
    \newline
    \newline
    \newline
    \newline
    \newline
\end{frame}

\begin{frame}
    % Izrek nam poda kriterij za določanje iracionalnosti nekaterih decimalnih števil.

    Izrek odpove za $Dec\{k^2\}$, saj $\sum_{k=1}^{\infty}\frac{1}{k^2}$ konvergira.
    \pause
    % V nadaljevanju želimo poiskati močnejši kriterij, ki nam bo povedal ali je dec št. iracionalno.

    \begin{izrek}
        Naj bo $Dec\{a_k\} \in \Q$. Potem obstaja $x \in \R, \ x > 1$ in pozitivna konstanta $C$,
        da velja $a_k \geq Cx^k$ za vsak $k \geq 1$.
    \end{izrek}
    \pause
    % Ta izrek nam pove, da če je decimalno št. zgrajeno iz blokov zaporedja a_k racionalno
    % potem zaporedje a_k narašča vsaj eksponentno.

    % Izreka ne bomo dokazali, če pa vas zanima dokaz si ga lahko pogledate na spletni učilnici
    % kjer bo objavljen v moji seminarski.

    % Bolj kot izrek pa nas bo zanimala naslednja posledica
    \begin{posledica}\label{posledica}
        Predpostavimo, da velja
        \[
            \sum_{k=1}^{\infty}\frac{y^k}{a_k} = \infty\]
        za vsak $y > 1$. Potem je decimalno število $Dec\{a_k\}$ iracionalno.
    \end{posledica}

\end{frame}

\begin{frame}
    % Za to posledico velja naslednje
    Posledica je močnejša kot izrek \ref{izrek1clanek1}:
    \begin{itemize}
        \item Pogoj iz izreka \ref{izrek1clanek1} 
        $(\sum_{i=1}^{\infty} \frac{1}{a_i} = \infty)$ očitno implicira
        pogoj iz zgornje posledice \ref{posledica}
        $(\sum_{k=1}^{\infty}\frac{y^k}{a_k} = \infty ,$ za vsak $y > 1)$.
        \pause
        \item Izreka \ref{izrek1clanek1} ne moremo uporabiti za zaporedja $\{a_k\}$, ki naraščajo kot $e^{\sqrt{k}}$,
                saj vrsta $\sum_{k=1}^{\infty}\frac{1}{a_k}$ konvergira.
            % uporabimo integralski kriterij in preverimo konvergenco za
            \pause
        \item Pogoj iz posledice \ref{posledica} je izpoljen za $Dec\{a_k\}$, kjer $\{a_k\}$ naraščajo kot $e^{\sqrt{k}}$.
            % uporabimo korenski kriterij in preverimo konvergenco
        \newline
        \newline
        \newline
        \newline
    \end{itemize}
\end{frame}

\begin{frame}
    \begin{zgled}
        S pomočjo posledice lahko preverimo ali je $Dec\{k^2\} = 0,1491625\dots$ iracionalno.
        \pause
        
        Preveriti moramo ali je \[\sum_{k=1}^{\infty}\frac{y^k}{k^2} = \infty; \ \forall y > 1\]
        \newline
        \newline
        \newline
        \newline
        \newline
        \newline
    \end{zgled}
    % Tu uporabiš kvocientni kriterij in dokažeš
    

\end{frame}

\begin{frame}
    \emph{Dokaz posledice:}

    dokazujemo: $\sum_{k=1}^{\infty}\frac{y^k}{a_k} = \infty; \ \forall y > 1 \ \Rightarrow Dec\{a_k\} \in \R \setminus \Q$
    \newline
    \newline
    \newline
    \newline
    \newline
    \newline
    \newline
    \newline
    \newline
    \newline
    \newline
    \newline
    \newline
\end{frame}

\begin{frame}
    \frametitle{Vprašanja}
\end{frame}



\end{document}