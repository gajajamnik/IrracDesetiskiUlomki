\documentclass{beamer}

\usepackage[slovene]{babel}
\usepackage{amsfonts,amssymb}
\usepackage[utf8]{inputenc}
\usepackage{lmodern}
\usepackage[T1]{fontenc}

\usetheme{Warsaw}

\def\N{\mathbb{N}} % mnozica naravnih stevil
\def\Z{\mathbb{Z}} % mnozica celih stevil
\def\Q{\mathbb{Q}} % mnozica racionalnih stevil
\def\R{\mathbb{R}} % mnozica realnih stevil
\def\C{\mathbb{C}} % mnozica kompleksnih stevil


\def\qed{$\hfill\Box$}   % konec dokaza
\newtheorem{izrek}{Izrek}
\newtheorem{trditev}{Trditev}
\newtheorem{posledica}{Posledica}
\newtheorem{lema}{Lema}
\newtheorem{definicija}{Definicija}
\newtheorem{pripomba}{Pripomba}
\newtheorem{primer}{Primer}
\newtheorem{zgled}{Zgled}
\newtheorem{zgledi}{Zgledi uporabe}
\newtheorem{zglediaf}{Zgledi aritmetičnih funkcij}
\newtheorem{oznaka}{Oznaka}

\title{Aritmetične funkcije}
\author{Marko Petkovšek}
\institute{Fakulteta za matematiko in fiziko \\
Oddelek za matematiko}
\date{24.\ februar 2017}

\begin{document}


%%%%%%%%%%%%%%%%%%%%%%%%%%%%%%%%%%%%%%%%%%%%%%%%%%%%%%%%%%%%%%%%%%%%%

\begin{frame}
\titlepage
\end{frame}

%%%%%%%%%%%%%%%%%%%%%%%%%%%%%%%%%%%%%%%%%%%%%%%%%%%%%%%%%%%%%%%%%%%%%

\begin{frame}

\begin{oznaka}
\[
\N = \{1,2,3,\ldots\}
\]
%\vspace{0.1pt}
\end{oznaka}

\pause
\medskip
\begin{definicija}
\alert{Aritmetična funkcija} je preslikava oblike
\[
f: \N \to A, \quad A \subseteq \C.
\]
\pause
Aritmetična funkcija $f$ je \alert{multiplikativna}, če za poljubni tuji števili $a, b \in \N$ velja:
\[
f(ab) = f(a)f(b).
\]
%\vspace{0.1pt}
\end{definicija}

\end{frame}

%%%%%%%%%%%%%%%%%%%%%%%%%%%%%%%%%%%%%%%%%%%%%%%%%%%%%%%%%%%%%%%%%%%%%

\begin{frame}

\begin{zglediaf}
\begin{enumerate}
\item $\tau(n) = $ število pozitivnih deliteljev števila $n$
\pause
\item $\sigma(n) = $ vsota pozitivnih deliteljev števila $n$
\end{enumerate}
\end{zglediaf}

\pause
\medskip
\begin{zgled}
\[
\begin{array}{c|c|cc}
n & {\rm pozitivni\ delitelji\ } n & \tau(n) & \sigma(n) \\
\hline
1 & 1 & 1 & 1 \\
2 & 1,2 & 2 & 3 \\
3 & 1,3 & 2 & 4 \\
4 & 1,2,4 & 3 & 7 \\
5 & 1,5 & 2 & 6 \\
6 & 1,2,3,6 & 4 & 12
\end{array}
\]
\end{zgled}

\pause
\medskip
\begin{trditev}
Funkciji $\tau$ in $\sigma$ sta multiplikativni.
\end{trditev}

\end{frame}


%%%%%%%%%%%%%%%%%%%%%%%%%%%%%%%%%%%%%%%%%%%%%%%%%%%%%%%%%%%%%%%%%%%%%

\begin{frame}
\frametitle{Eulerjeva funkcija}

\pause
\begin{definicija}
Za vse $n \in \N$ s \alert{$\varphi(n)$} označimo število 
celih števil iz množice $\{1, 2, \ldots, n\}$, ki so tuja številu $n$.
Preslikavo \alert{$\varphi: \N \rightarrow \N$} imenujemo \alert{Eulerjeva funkcija}.
\end{definicija}
\pause

\vfill
\begin{zgled}
\[
\begin{array}{clc}
 n & \{1, 2, \ldots, n\}          & \varphi(n)       \\
 \hline
 1 & \{\bf{1}\}                    &     1      \\
 2 & \{{\bf 1},2 \}               &     1      \\
 3 & \{{\bf 1,2},3 \}             &     2      \\
 4 & \{{\bf 1},2,{\bf 3},4 \}     &     2      \\
 5 & \{{\bf 1,2,3,4},5 \}         &     4      \\
 6 & \{{\bf 1},2,3,4,{\bf 5},6 \} &     2
\end{array}
\]
\end{zgled}

\end{frame}


%%%%%%%%%%%%%%%%%%%%%%%%%%%%%%%%%%%%%%%%%%%%%%%%%%%%%%%%%%%%%%%%%%%%%

\begin{frame}

{\bf Vprašanje:} \quad $\varphi(10^{10}) = \varphi(10000000000) =\ ?$
\pause

\bigskip
\begin{trditev}
Naj bo $p$ praštevilo. Potem je $\varphi(p) = \pause p-1$.
\end{trditev}
\pause

\bigskip
\begin{trditev}
Naj bo $p$ praštevilo in $k \in \N$. Potem je $\varphi(p^k) = \pause p^k-p^{k-1}$.
\end{trditev}
\pause

\bigskip
\begin{izrek}
Če sta $a$ in $b$ tuji naravni števili, je
$\varphi(a b) = \varphi(a)\varphi(b)$.
\end{izrek}


\end{frame}


%%%%%%%%%%%%%%%%%%%%%%%%%%%%%%%%%%%%%%%%%%%%%%%%%%%%%%%%%%%%%%%%%%%%%

\begin{frame}

\begin{posledica}
\[
\varphi(n)\ =\ n\times \prod_{p\,|\,n} \left(1 - \frac{1}{p}\right)
\]
\end{posledica}

\bigskip
\pause
\begin{izrek}
\[
\sum_{d\,|\,n} \varphi(d)\ =\ \pause n
\]
\end{izrek}


\end{frame}

%%%%%%%%%%%%%%%%%%%%%%%%%%%%%%%%%%%%%%%%%%%%%%%%%%%%%%%%%%%%%%%%%%%%%

\begin{frame}
\begin{izrek}[Eulerjev izrek]
Naj bosta $n \in \N$ in $a \in \Z$ tuji števili. Potem je
\[
a^{\varphi(n)} \equiv 1 \pmod{n}.
\]
\end{izrek}

\pause

\bigskip
\begin{posledica}[mali Fermatov izrek]
Naj bo $p$ praštevilo in $a \in \Z$ celo število, ki ni deljivo s $p$. Potem je
\[
a^{p-1} \equiv 1 \pmod{p}.
\]
\end{posledica}


\end{frame}

%%%%%%%%%%%%%%%%%%%%%%%%%%%%%%%%%%%%%%%%%%%%%%%%%%%%%%%%%%%%%%%%%%%%%

\begin{frame}
\frametitle{M\"obiusova funkcija}

\pause
\begin{definicija}
Preslikavo \alert{$\mu: \N \rightarrow \Z$}, definirano s predpisom
\[
\mu(n)\ =\ \left\{
\begin{array}{cl}
0, & \mbox{če\ } n \mbox{\ deljiv s kvadratom praštevila,} \\
(-1)^r, & \mbox{sicer,}
\end{array}
\right.
\]
kjer je $r$ število različnih prafaktorjev števila $n$, imenujemo \em{M\"obiusova funkcija}.
\end{definicija}

\pause
\[
\begin{array}{c|*{10}{r}}
   n   & 1 & 2 & 3 & 4 & 5 & 6 & 7 & 8 & 9 & 10 \\
\hline
\mu(n) & \ \ 1 & -1 & -1 & \ \ 0 & -1 & \ \ 1 & -1 & \ \ 0 & \ \ 0 & \ \ 1
\end{array}
\]
\end{frame}

%%%%%%%%%%%%%%%%%%%%%%%%%%%%%%%%%%%%%%%%%%%%%%%%%%%%%%%%%%%%%%%%%%%%%

\begin{frame}

\bigskip

\begin{izrek}
Če sta $a$ in $b$ tuji naravni števili, je
$\mu(a b) = \mu(a)\mu(b)$.
\end{izrek}

\pause

\bigskip
\begin{trditev}
Za vse $n \in \N$ velja enačba
\[
\label{mirek}
\sum_{d\,|\,n} \mu(d)\ =\ \left\{
\begin{array}{ll}
1, & n = 1, \\
0, & n > 1,
\end{array}
\right.
\]
kjer $d$ preteče vse pozitivne delitelje števila $n$.
\end{trditev}

\pause

\bigskip
\begin{posledica}
\[
\mu(n)\ =\ \left\{
\begin{array}{cl}
1, & n = 1, \\
-\displaystyle\sum_{d\,|\,n, \,d < n} \mu(d), & n > 1.
\end{array}
\right.
\]
\end{posledica}

\end{frame}

%%%%%%%%%%%%%%%%%%%%%%%%%%%%%%%%%%%%%%%%%%%%%%%%%%%%%%%%%%%%%%%%%%%%%

\begin{frame}

\begin{izrek} \alert{(M\"obiusov obrat)} \ 
Za aritmetični funkciji $f, g: \N \rightarrow \C$ velja:
\[
g(n)\ =\ \sum_{d\,|\,n} f(d)\quad \Longleftrightarrow\quad f(n)\ =\ \sum_{d\,|\,n} \mu\left(\frac{n}{d}\right)g(d)
\]
\end{izrek}

\pause
\bigskip
\begin{zgledi}
\begin{eqnarray*}
\sum_{d\,|\,n}\varphi(d)\ =\ n\quad &\Longrightarrow&\quad\pause \varphi(n)\ =\ \sum_{d\,|\,n}\mu\left(\frac{n}{d}\right)d \\
\pause
\tau(n)\ =\ \sum_{d\,|\,n} 1 \quad &\Longrightarrow& \quad\pause \sum_{d\,|\,n}\mu\left(\frac{n}{d}\right)\tau(d)\ =\ 1 \\
\pause
\sigma(n)\ =\ \sum_{d\,|\,n} d \quad &\Longrightarrow& \quad\pause \sum_{d\,|\,n}\mu\left(\frac{n}{d}\right)\sigma(d)\ =\ n
\end{eqnarray*}
\end{zgledi}


\end{frame}


\end{document}