\documentclass[a4paper,12pt]{article}

\usepackage[slovene]{babel}
\usepackage{amsfonts,amssymb,amsmath}
\usepackage[utf8]{inputenc}
\usepackage[T1]{fontenc}
\usepackage{lmodern}
\usepackage{graphicx}

\usepackage{url}
\usepackage{icomma}


\def\N{\mathbb{N}} % mnozica naravnih stevil
\def\Z{\mathbb{Z}} % mnozica celih stevil
\def\Q{\mathbb{Q}} % mnozica racionalnih stevil
\def\R{\mathbb{R}} % mnozica realnih stevil
\def\C{\mathbb{C}} % mnozica kompleksnih stevil
\newcommand{\geslo}[2]{\noindent\textbf{#1} \quad \hangindent=1cm #2\\[-1pc]}

\def\qed{$\hfill\Box$}   % konec dokaza
\def\qedm{\qquad\Box}   % konec dokaza v matematičnem načinu
\newtheorem{izrek}{Izrek}
\newtheorem{trditev}{Trditev}
\newtheorem{posledica}{Posledica}
\newtheorem{lema}{Lema}
\newtheorem{opomba}{Opomba}
\newtheorem{definicija}{Definicija}
\newtheorem{zgled}{Zgled}

\title{O nekaterih iracionalnih desetiških ulomkih \\ 
\Large Seminar}
\author{Gaja Jamnik \\
Fakulteta za matematiko in fiziko \\
Oddelek za matematiko}
\date{2.\ april 2021}

\begin{document}

%%%%%%%%%%%%%%%%%%%%%%%%%%%%%%%%%%%%%%%%%%%%%%%%%%%%%%%%%%%%%%%%%%%%%


\maketitle


%%%%%%%%%%%%%%%%%%%%%%%%%%%%% UVOD %%%%%%%%%%%%%%%%%%%%%%%%%%%%%%%%%%%%%%%%%%%%%%%

\section{Uvod}
V tem dokumentu bomo obravnavali realna števila zapisana v obliki decimalnega zapisa.
Spomnimo se definicije desetiških ulomkov ter decimalnega zapisa.

\begin{definicija}
    {\em Desetiški ulomek} je ulomek, katerega imenovalec je potenca števila $10$. Zapis desetiškega ulomka navadno
    nadomestimo z zapisom z decimalno vejico, ki mu pravimo {\em decimalni zapis}. Če ponazorimo s primerom, namesto 
    $ \frac{23}{1000}$ pišemo $0,023$.
\end{definicija}

Pri obravnavi številskih množic se že v srednji šoli spoznamo z razliko v decimalnem zapisu
racionalnih in iracionalnih števil. Racionalna števila lahko zapišemo
s končnim ali neskončnim periodičnim decimalnim zapisom, medtem ko je zapis iracionalnega števila možen 
le z neskončno neperiodičnimi decimalkami.

Včasih pa iz zapisa decimalnega števila, z neskončno števkami ne moremo razbrati ali je periodično on nekega člena dalje.
Zanimalo nas bo, kaj nam lahko v takem primeru decimalni zapis realnega števila $x$ pove o njegovi racionalnosti
oz.\ iracionalnosti.
Zaradi poenostavitve privzemimo, da velja $ 0 < x < 1$.

V ta namen podajmo naslednjo definicijo.

\begin{definicija}
    \label{defincija blokov}
    Naj bo $x$ realno število, $ 0 < x < 1$, podano z decimalnim zapisom: 
    \[ 
        \begin{split}
            x & = \sum^n_{i=1} c_i 10^{-i} = \\
            & = 0,c_0c_1c_2 \cdots c_n ,
        \end{split}   
    \]
    kjer so $0 \leq c_i \leq 9 \ \forall i = 1, \dots, s$.
    
    Z $b$ označimo celo število sestavljeno iz zaporedja števk
    $b_1b_2b_3 \dots b_s$, 
    kjer je $s\geq 1$ in $0 \leq b_i \leq 9 \ \forall i = 1, \dots, s$.
    Pravimo, da število $x$ {\em vsebuje blok števil $(b) = (b_1b_2b_3 \dots b_s)$}, če obstaja $j \geq 0$, da je 
    $c_{i+j} = b_i$ za vse $i=1, 2, \dots s$.
\end{definicija}

\begin{zgled}
    Število $0,135627$ vsebuje blok $(356)$, vendar ne vsebuje bloka $(352)$.
\end{zgled}

Poljubno decimalno število lahko razumemo kot zaporedje blokov celih števil.
Tako je na primer število $0,11223344 \cdots$ zaporedje blokov $(nn)$,
kjer je $n \in \N$.

%%%%%%%%%%%%%%%%%%%%%%%%%% O ŠTEVILU 0,23571113 %%%%%%%%%%%%%%%%%%%%%%%%%%%%%%%%%%%%%%%%%%%%%%%%%

\section{O številu $0,23571113 \dots$}

Za začetek obravnavajmo število $0,23571113 \dots$, ki je zgrajeno iz zaporedja
blokov praštevil. 
Pri algebri smo letos dokazali Evklidov izrek, ki pravi, da je praštevil neskončno mnogo,
zato to decimalno število ne bo končno. 
Iz zapisa pa ne moremo razbrati ali 
je po nekem členu decimalno število periodično, zato ni očitno ali je število racionalno.
Trdimo naslednje:

\begin{trditev}
    \label{trditev praštevila}
    Število $0,23571113 \dots$ je iracionalno.
\end{trditev}

Za dokaz te trditve bomo potrebovali različico Dirichletovega izreka \cite{Vog}, ki pravi naslednje:
% VSTAVI SKLIC

\begin{izrek}[Dirichletov izrek]
    \label{Dirichletov izrek}
    V vsakem zaporedju $ \lbrace an + b \rbrace_{n \in \N_0}$ naravnih števil, kjer sta $a$ in $b$
    tuji si naravni števili, je neskončno praštevil.
\end{izrek}

\noindent
{\em Dokaz trditve:\/} Naj bo $s \geq 0$ celo število. Po izreku \ref{Dirichletov izrek} vsako zaporedje
$ \{10^{s+1}k + 1\}$, $k \in \N$ vsebuje neskončno praštevil. Torej obstajajo praštevila
oblike $(k)\underbrace{00 \dots}_{s}1$, kjer števkam števila $k$ sledi $s$ ničel ter ena enica. 

Decimalno število $0,23571113\dots$ očitno vsebuje vse bloke take oblike za vsak $s \geq 0$. Z večanjem števila
$s$ narašča tudi število ničel v posameznem bloku, kar pomeni, da po še tako pozni decimalki zapis ne bo periodičen.

\qed


%%%%%%%%%%%%%%%%%%% DECIMALNA ŠT Z NARAŠČAJOČIMI BLOKI %%%%%%%%%%%%%%%%%%%%%%%%%%%%%%%%%%%%%%%%%%%%%

\section{Decimalna števila z naraščajočimi bloki}

V prejšnjem razdelku smo dokazali, da je decimalno število sestavljeno iz praštevilskih
blokov iracionalno. Kaj pa lahko povemo za decimalno število sestavljeno iz poljubnih blokov?

Naj bo $1 \leq a_1 < a_2 < \dots $ strogo naraščajoče zaporedje celih števil. 
Označimo: \[Dec\{a_k\} = 0,(a_1)(a_2)(a_3)... \  \ ;  a_k \in \Z \ k \in \N. \]
Zanimajo nas lastnosti zaporedja $\{a_k\}$, ki nam zagotovijo, da bo $Dec\{a_k\}$ iracionalno.


\begin{izrek}\label{irac1}
    
    Če za strogo naraščajoče zaporedje celih števil $\{a_i\}_{i \in \N}$ velja 
    \[ \sum_{i=1}^{\infty} \frac{1}{a_i} = \infty ,\]
    potem je $Dec\{a_k\}$ iracionalno.
\end{izrek}

Ta izrek je posplošitev trditve \ref{trditev praštevila}, ki pravi, da je število $0,23571113 \dots$
iracionalno. Za zaporedje praštevil $2, 3, 5, 7, \dots$ namreč velja
\[
    \sum_{p \ \text{praštevilo}}\frac{1}{p} = \frac{1}{2} + \frac{1}{3} + \frac{1}{5} + \frac{1}{7} + \dots = \infty ,\]
kar je leta 1737 dokazal Leonhard Euler v \cite{Eul}.

%% ali je ta zgled primeren?
\begin{zgled}
    Decimalno število $Dec\{7n\} = 0,714212835\dots (7n) \dots$ je iracionalno, saj
    bo vrsta $\sum_{k=1}^{\infty}\frac{1}{7n} $ divergirala.
\end{zgled}

Za dokaz izreka \ref{irac1} bomo potrebovali naslednjo lemo.

\begin{lema}
    Naj bo $(b) = (b_1b_2b_3 \dots b_s)$ blok števil. Z $X = X(b_1b_2b_3 \dots b_s)$ označimo
    zaporedje naravnih števil, ki ne vsebujejo bloka števil (b). Potem 
    \[ \sum_{x \in X}^{\infty} \frac{1}{x}\] konvergira.
\end{lema}

\begin{opomba}
    \label{lema bloki}
    V lemi, v nasprotju z definicijo \ref{defincija blokov}, obravnavamo vsebovanost blokov v celih
    in ne decimalnih številih. Definicijo v ta namen prilagodimo tako, da le spremenimo potenco števila
    $10$ v vsoti. Celo število $x$ tako zapišemo kot 
    \[ 
        \begin{split}
            x & = \sum^n_{i=1} c_i 10^{n-i} = \\
            & = c_0c_1c_2 \cdots c_n \ .
        \end{split} \]
\end{opomba}

%------------------------ DOKAZ LEME --------------------------------------------
\noindent
{\em Dokaz leme \ref{lema bloki}:\/} 
Označimo delno vsoto naše vrste 
\[ S_n = \frac{1}{x_1} + \frac{1}{x_2} + \frac{1}{x_3} + \dots + \frac{1}{x_n} \ . \]
Vzemimo tak $t \in \N $, da velja $x_{t-1} < 10^s \leq x_t$. Število števk bloka $(b) = (b_1b_2b_3 \dots b_s)$
je očitno $s$, število $10^s$ pa ima $s+1$ števk, kar pomeni, da imajo vsi $x_i$ za $i \geq t$
več števk kot je dolžina bloka. 

Delno vsoto sedaj preoblikujemo v
\[ S_n = \frac{1}{x_1} + \dots \frac{1}{x_t} + 10^{-s}(\frac{1}{\frac{x_{t + 1}}{10^s}} + \dots + \frac{1}{\frac{x_n}{10^s}})
    \leq \frac{1}{x_1} + \dots \frac{1}{x_t} + 10^{-s}(\frac{1}{ \lfloor\frac{x_{t + 1}}{10^s}\rfloor} + \dots + \frac{1}{\lfloor\frac{x_n}{10^s}\rfloor}) ,
\]
kjer smo v neenakosti upoštevali, da za poljuben $y > 0$ velja $y \geq \lfloor y \rfloor$.

Ker za vsak $\ t < i \leq n$ velja $10^s \leq x_i$, bo $\lfloor \frac{x_i}{10^s} \rfloor$ pozitivno celo število.
To število lahko interpretiramo kot $x_i$ brez zadnjih $s$ števk. Ker vsi $x_i \in X$
ne vsebujejo bloka $(b)$, ga očitno ne vsebuje niti njegovih prvih nekaj števk. 
Od tod sledi, da za vsak $x_i, \ t < i \leq n$ obstaja $x_y \in X$,
tako da $ x_y = \lfloor \frac{x_i}{10^s} \rfloor$.

Novo nastala števila, pa so si lahko med seboj enaka. Tako bi na primer za $s=2$ veljalo 
$ \lfloor\frac{12345}{10^2}\rfloor = \lfloor \frac{12387}{10^2}\rfloor = 123$.

Opazimo, da se blok $(b_1b_2b_3 \dots b_s)$ pojavi v vsaj eni od prvih $10^s$ naravnih števil.
Torej lahko poljuben $x_y \in X$ zadošča za največ $10^s - 1$ možnih $x_i$, kjer $\lfloor \frac{x_i}{10^s} \rfloor = x_y$.
V zgornjem primeru (za $s=2$) torej število $123$ zadošča za vsa števila $123 \square \square$
kjer lahko na zadnji dve mesti postavimo poljubno dvomestno število, razen bloka $b_1b_2$.
Takih števil pa je največ $10^2 - 1$.

Po zgornjem premisleku lahko sedaj ocenimo izraz 
\[(\frac{1}{ \lfloor\frac{x_{t + 1}}{10^s}\rfloor} + \dots + \frac{1}{\lfloor\frac{x_n}{10^s}\rfloor}) < (10^s - 1)S_n.\]
S pomnožitvijo $(10^s - 1)$ s $S_n$ smo tako zagotovo zajeli vse člene na levi strani neenakosti
z upoštevanjem vseh možnih ponovitev v vrednostih $\lfloor\frac{x_i}{10^s}\rfloor$.
Če to oceno uporabimo na delni vsoti, dobimo:
\[
    \begin{split}
    S_n &< \sum_{i=1}^t \frac{1}{x_i} + (10^s - 1)10^{-s}S_n \\
    S_n &< 10^s \sum_{i=1}^t \frac{1}{x_i}
    \end{split}
    \]

%% BOLJŠE RAZLOŽI TALE SKLEPP!!!
Desna stran neenačbe je neodvisna od $n$, zato vrsta konvergira.

\qed

% ---------------------- DOKAZ IZREKA 1 ---------------------------------------
\noindent
{\em Dokaz izreka \ref{irac1}:\/} Naj bo $\{a_i\}_{i\in \N}$ strogo naraščajoče zaporedje
celih števil za katero velja $\sum_{i=1}^{\infty}\frac{1}{a_i} = \infty$. Izrek bomo dokazali
s protislovjem. Predpostavimo, da je $Dec\{a_i\} = 0,(a_1)(a_2) \dots \in \Q$. To decimalno
število očitno ni končno, torej je periodično. 
To pomeni, da obstaja nek blok števil $(b) = (b_1b_2b_3 \dots b_s)$,
ki se v decimalnem zapisu periodično ponavlja od nekega mesta dalje. 

Definirajmo tak blok $(c)$ dolžine $2s$, da velja: če je $(b) = (11 \cdots 1)$, naj bo blok $(c)$
iz samih dvojic, v nasprotnem primeru pa naj bo $(c)$ iz samih enic.

Naj bo $Y = Y(c_1, c_2, \dots, c_{2s})$ zaporedje naravnih števil, ki ne vsebuje bloka $(c)$.
Razdelimo našo vsoto glede na vsbovanost členov $a_i$ v $Y$:
\[
    \sum_{i=1}^{\infty} \frac{1}{a_i} = \sum_{a_i \in Y}\frac{1}{a_i} + \sum_{a_i \notin Y} \frac{1}{a_i}
    \]

Po predpostavki vsota na levi strani enačaja divergira in po lemi vsota $\sum_{a \in Y}\frac{1}{a}$
konvergira. Od tod sledi, da bo $\sum_{a \notin Y} \frac{1}{a}$ divergirala.
To pomeni, da bo obstaja neskončno $a_i$ v $Dec\{a_i\} = 0,(a_1)(a_2) \dots$, ki vsebujejo blok 
$(c_1c_2\dots c_{2s})$. Ta blok je dvakrat daljši kot blok $(b)$ in z drugačnimi števkami, zato se ne more
zgoditi, da bi bil $(c)$ vsebovan v $(b)$ ali sestavljal njegove dele. 
%% Tj. ne more se zgoditi, da bi na primer
Blok $(b)$ se zato ne bo ponavljal v neskončnosti in posledično $Dec\{a_i\}$ ne more biti periodično.

\qed

%%%%%%%%%%%%%%%%%%%%%%%%%%%%%%%%%%%%%%%%%%%%%%%%%%%%%%%%%%%%%%%%%%%%%%%%%%%%%%%

Dokazan izrek nam poda kriterij iracionalnosti števila $Dec\{a_k\}$,
ki pa odpove za marsikatero zaporedje $\{a_k\}$.
Tako na primer kriterij ne pove nič o iracionalnosti števila $Dec\{k^2\}$, saj vrsta $\sum_{k=1}^{\infty}\frac{1}{k^2}$ konvergira.

Predpostavimo, da zaporedje $a_k$ narašča kot $e^{\sqrt{k}}$.
Z uporabo integralskega kriterija za konvergenco vrst lahko preverimo, da pogoj izreka \ref{irac1}
za tako zaporedje ne bo izpolnjen
($\frac{1}{e^{\sqrt{x}}}$ je zvezna, pozitivna in padajoča na $[1, \infty)$ zato
integralski kriterij lahko uporabimo):
\[
    \begin{split}
    \int_1^{\infty}\frac{1}{e^{\sqrt{x}}}dx &= 2e^{-\sqrt{x}}(- \sqrt{x} - 1) \big|_1^{\infty} = \\
    &= -\frac{4}{e}
    \end{split}
\]
Zgornji integral konvergira, zato bo vrsta $\sum_1^{\infty}\frac{1}{e^{\sqrt{x}}}$ konvergentna
in s tem tudi $\sum_1^{\infty}\frac{1}{a_k}$.

V nadaljevanju bomo pokazali, da je $Dec\{a_k\}$ iracionalno, tudi kadar $\{a_k\}$ narašča kot $e^{\sqrt{k}}$.

\begin{izrek}
    \label{izrek1clanek2}
    Naj bo $Dec\{a_k\} \in \Q$. Potem obstaja $x \in \R, \ x > 1$ in pozitivna konstanta $C$,
    da velja $a_k \geq Cx^k$ za vsak $k \geq 1$.
\end{izrek}

Izrek pove, da če je $Dec\{a_k\}$ racionalno, potem zaporedje $a_k$ narašča vsaj eksponentno.

\begin{posledica}
    \label{posledica}
    Predpostavimo, da velja
    \[
        \sum_{k=1}^{\infty}\frac{y^k}{a_k} = \infty\]
    za vsak $y > 1$. Potem je decimalno število $Dec\{a_k\}$ iracionalno.
\end{posledica}

Zgornja posledica je močnejša kot izrek \ref{irac1}. Pogoj iz izreka \ref{irac1} 
$(\sum_{i=1}^{\infty} \frac{1}{a_i} = \infty)$ očitno implicira
pogoj iz zgornje posledice $(\sum_{k=1}^{\infty}\frac{y^k}{a_k} = \infty , \ \forall y > 1)$.
Po drugi strani pa bo iz posledice sledilo, da so $Dec\{a_k\}$ iracionalna tudi za zaporedja
$\{a_k\}$, ki naraščajo kot $e^{\sqrt{k}}$, oz. kot $e^{k^s}$ za $0 < s < 1$:

S korenskim kriterijem preverimo konvergenco.
\[
    r = \lim_{k \rightarrow \infty}\sqrt[k]{\frac{y^k}{e^{\sqrt{k}}}} = \lim_{k \rightarrow \infty}\frac{y}{e^{\frac{1}{\sqrt{k}}}}
    = y
\]
Po predpostavki iz posledice je $y > 1$, zato vrsta divergira in bo zato $Dec\{a_k\}$ iracionalno.
\\

\begin{zgled}
    S pomočjo posledice \ref{posledica} lahko sedaj preverimo ali je $Dec\{k^2\}$ iracionalno.
    Po posledici bo število $0,149162536\dots$ iracionalno, če bo $\sum_{k=1}^{\infty}\frac{y^k}{k^2} = \infty$
    za vsak $y > 1$. Uporabimo kvocientni kriterij in preverimo konvergenco vrste.
    \[
        L = \lim_{k \rightarrow \infty} \frac{\frac{y^{k+1}}{(k+1)^2}}{\frac{y^k}{k^2}}
        = y \lim_{k \rightarrow \infty} \big (\frac{k}{k + 1}\big)^2 = y > 1
    \]
    Vrsta divergira, zato bo $Dec\{k^2\} \in \R \setminus \Q$.
\end{zgled}

\noindent
{\em Dokaz posledice \ref{posledica}:\/}
Dokazujemo s protislovjem. Predpostavimo, da je $Dec\{a_k\} \in \Q$.
Po izreku \ref{izrek1clanek2} obstaja realno število $x > 1$ in konstanta $C>0$, da velja
$a_k \geq Cx^k$ za vsak $k \geq 1$. Ta izraz preoblikujemo in dobimo:
\[
    \frac{\sqrt{x}^k}{a_k} \leq \frac{1}{C\sqrt{x}^k}\] 
Od tod sledi
\[ \sum_{k=1}^{\infty} \frac{\sqrt{x}^k}{a_k} \leq C^{-1} \sum_{k=1}^{\infty} \frac{1}{\sqrt{x}^k} < \infty.\]
Desna vsota je ravno geometrijska vrsta, ki konvergira saj je $x > 1$ in s tem $\sqrt{x} > 1$.
To pa je v protislovjem s predpostavko, ki pravi da $\sum_{k=1}^{\infty}\frac{y^k}{a_k} = \infty$ za vsak $y>1$.
Torej bo $Dec\{a_k\}$ iracioanlno.
\qed

{\em Dokaz izreka \ref{izrek1clanek2}:\/} Predpostavimo, da $Dec\{a_k\} = 0,(a_1)(a_2)\dots \in \Q$, torej bo 
periodično od nekega člena dalje. Naj bo
$(b_1b_2 \dots b_p), \ 0 \leq b_i \leq 9 \ \forall i = 1, \dots 9$, perioda in naj bo
$(a_m)$ prvi blok, ki se pojavi v periodičnem delu decimalnega zapisa.

Najprej dokažimo, da ima blok $(a_{k+p})$ vsaj eno števko več kot blok $(a_k)$ za $\forall k \geq m$.
Predpostavimo, da imata $(a_{k+p})$ in $(a_k)$ oba po $N$ števk. 
Ker je zaporedje $\{a_k\}$ strogo naraščajoče bodo imeli vsi $a_k, a_{k+1}, \dots a_{k+p}$ po $N$ števk.
Število $(a_k)(a_{k+1})\dots (a_{k+p-1})$ ima torej $Np$ števk. Ker pa je perioda enaka p,
bo od tod sledilo, da je $a_k = a_{k+p}$, kar ne more biti res, saj je $\{a_k\}$ strogo naraščajoče zaporedje.

Sledi, da ima $a_{k+ 2p}$ vsaj eno števko več kot $a_{k+p}$, in zato vsaj dve števki več kot $a_k$
za vsak $k \geq m$.
Torej velja, da je $a_{k+2p} \geq 10 a_k$. Z indukcijo lahko dokažemo, da je 
\[ a_{k+2np} \geq 10^n a_k; \ \forall k \geq m, \ n \in \N.\]

Vzemimo sedaj poljuben $l \geq m$ in označimo $n := \lfloor \frac{l-m}{2p} \rfloor$.
Veljalo bo 
\begin{equation}\label{enacba1}
    a_l \geq a_{m+2np} \geq 10^n a_m \geq 10 ^{\frac{l-m-2p}{2p}}a_m,
\end{equation}
za kar dokažimo vsak neenačaj posebej.

Za dokaz prve neenakosti si poglejmo indekse obeh členov. 
\[
    \begin{split}
    m + 2np &= m + 2 \bigg\lfloor \frac{l-m}{2p}\bigg\rfloor p \\
    &\leq m + 2\bigg(\frac{l-m}{2p}\bigg)p \\
    &= m + l - m = l
    \end{split}
    \] 
Ker je $\{a_k\}$ strogo naraščajoče bo sledilo $a_{m+2np} \leq a_l$.
Druga neenakostj sledi po prejšnji ugotovitvi, da je $a_{k+2np} \geq 10^n a_k$.
Za tretjo neenakost pa podrobneje poglejmo eksponenta števila $10$. Velja
\[
    n = \bigg\lfloor \frac{l-m}{2p}\bigg\rfloor > \frac{l-m}{2p} - 1,
     \]
saj je po pravilih za računanje s celim delom realnega števila $x < \lfloor x\rfloor + 1$.

Definirajmo $C' := 10^{-\frac{m + 2p}{2p}}$ in $x := 10^{\frac{1}{2p}}$, za kar bo veljalo
\begin{equation}\label{enacba2}
     C'x^l = 10^{-\frac{m + 2p}{2p}} \cdot 10^{\frac{l}{2p}} = 10 ^{\frac{l-m-2p}{2p}}
     \leq 10 ^{\frac{l-m-2p}{2p}} a_m,
\end{equation}
kjer smo upoštevali, da je $\{a_k\}$ zaporedje pozitivnih celih števil in je zato $a_m \geq 1$.

Če sedaj združimo neeančbi \ref{enacba1} in \ref{enacba2} dobimo 
$a_l \geq C'x^l$ za vsak $l \geq m$.
Po potrebi zmanšajmo $C'$ na $C>0$, da bo neenakost veljala
za vse $k \geq 1$.
%Torej smo dobili $x>1$ in $C>0$, da $a_k \geq Cx^k; \ \forall k \geq 1$.

\qed

\section*{Angleško-slovenski slovar strokovnih izrazov}

\geslo{decimal fraction}{desetiški ulomek}

\geslo{Dirichlet's theorem}{Dirichletov izrek}

\geslo{floor function}{celi del števila}

\geslo{integral test}{integralski kriterij}

\geslo{iracioanl number}{iracionalno število}

\geslo{prime number}{praštevilo}

\geslo{root test}{korenski kriterij}

\geslo{racional number}{racionalno število}

\geslo{sequence}{zaporedje}

\begin{thebibliography}{1}
    \bibitem{Eul}
    L.~Euler, \emph{Variae observationes circa series infinitas},
    Commen.~Academ.~Scient.~Petropolitanae \textbf{9} (1737), 160-188.

    \bibitem{HaW}
    G.~H.~Hardy in E.~M.~Wright, \emph{An introduction to the theory of numbers}, 
    5.izdaja, Clarendon Press, Oxford, 1970.
    \bibitem{Heg}
    N.~Hegyvari, \emph{On some irrational decimal fractions},
    Amer.~Math.~Monthly \textbf{100(8)}  (Okt., 1993),  779-780.
    \bibitem{Mar}
    P.~Martinez, \emph{Some new irrational decimal fractions},
    Amer.~Math.~Monthly \textbf{108(3)}  (Mar., 2001),  250-253.
    \bibitem{Mer}
    A.~McD.~Mercer, \emph{A note on some irrational decimal fractions},
    Amer.~Math.~Monthly \textbf{101(6)}  (Jun.-Jul., 1994),  567-568.
    \bibitem{Vog}
    J.~Vogrinc, \emph{Dirichletov izrek in karakterizacija praštevil}, 
    magistrsko delo, Fakulteta za matematiko in fiziko, Univerza v Ljubljani, 2013.
    
\end{thebibliography}

\end{document}