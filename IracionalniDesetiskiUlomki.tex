\documentclass[a4paper,12pt]{article}

\usepackage[slovene]{babel}
\usepackage{amsfonts,amssymb,amsmath}
\usepackage[utf8]{inputenc}
\usepackage[T1]{fontenc}
\usepackage{lmodern}
\usepackage{graphicx}

\usepackage{url}


\def\N{\mathbb{N}} % mnozica naravnih stevil
\def\Z{\mathbb{Z}} % mnozica celih stevil
\def\Q{\mathbb{Q}} % mnozica racionalnih stevil
\def\R{\mathbb{R}} % mnozica realnih stevil
\def\C{\mathbb{C}} % mnozica kompleksnih stevil
\newcommand{\geslo}[2]{\noindent\textbf{#1} \quad \hangindent=1cm #2\\[-1pc]}

\def\qed{$\hfill\Box$}   % konec dokaza
\def\qedm{\qquad\Box}   % konec dokaza v matematičnem načinu
\newtheorem{izrek}{Izrek}
\newtheorem{trditev}{Trditev}
\newtheorem{posledica}{Posledica}
\newtheorem{lema}{Lema}
\newtheorem{opomba}{Opomba}
\newtheorem{definicija}{Definicija}
\newtheorem{zgled}{Zgled}

\title{O nekaterih iracionalnih desetiških ulomkih \\ 
\Large Seminar}
\author{Gaja Jamnik \\
Fakulteta za matematiko in fiziko \\
Oddelek za matematiko}
\date{2.\ april 2021}

\begin{document}

%%%%%%%%%%%%%%%%%%%%%%%%%%%%%%%%%%%%%%%%%%%%%%%%%%%%%%%%%%%%%%%%%%%%%


\maketitle


%%%%%%%%%%%%%%%%%%%%%%%%%%%%%%%%%%%%%%%%%%%%%%%%%%%%%%%%%%%%%%%%%%%%%

\section{Uvod}
V tem dokumentu bomo obravnavali realna števila zapisana v obliki decimalnega zapisa.
Spomnimo se definicije desetiških ulomkov ter decimalnega zapisa.

\begin{definicija}
    {\em Desetiški ulomek} je ulomek, katerega imenovalec je potenca števila $10$. Zapis desetiškega ulomka navadno
    nadomestimo z zapisom z decimalno vejico, ki mu pravimo {\em decimalni zapis}. Če ponazorimo s primerom, namesto 
    $ \frac{23}{1000}$ pišemo $0,023$ 
\end{definicija}

Pri obravnavi številskih množic se že v srednji šoli spoznamo z razliko v decimalnem zapisu
racionalnih in iracionalnih števil. Racionalna števila lahko zapišemo
s končnim ali periodičnim decimalnim zapisom, medtem ko je zapis iracionalnega števila možen 
le z neskončno neperiodičnimi decimalkami.

Zanimalo nas bo pa ravno obratno. Kaj nam lahko decimalni zapis realnega števila $x$ pove o njegovi racionalnosti
oz.\ iracionalnosti. \\
OPOMBA: Zaradi poenostavitve privzemimo, da velja $ 0 \le x \le 1$.

V ta namen podajmo naslednjo definicijo.

\begin{definicija}
    \label{defincija blokov}
    Naj bo $x$ realno število, $ 0 \le x \le 1$, podano z decimalnim zapisom: 
    \[ 
        \begin{split}
            x & = \sum^n_{i=1} c_i 10^{-i} = \\
            & = 0,c_0c_1c_2 \cdots c_n ,
        \end{split}   
    \]
    kjer so $0 \leq c_i \leq 9 \ \forall i = 1, \dots, s$.
    
    Z $b$ označimo celo število sestavljeno iz zaporedja števk
    $b_1b_2b_3 \dots b_s$, 
    kjer je $s\geq 1$ in $0 \leq b_i \leq 9 \ \forall i = 1, \dots, s$.
    Pravimo, da število $x$ {\em vsebuje blok števil $(b) = (b_1b_2b_3 \dots)$}, če obstaja $j \geq 0$, da je 
    $c_{i+j} = b_i$ za vse $i=1, 2, \dots s$.
\end{definicija}

\begin{zgled}
    Število $0,135627$ vsebuje blok $(356)$, vendar ne vsebuje bloka $(352)$.
\end{zgled}

Poljubno decimalno število lahko razumemo kot zaporedje blokov celih števil.
Tako je na primer število $0,11223344 \cdots$ zaporedje blokov $(nn)$,
kjer je $n \in \N$.

\section{O številu $0,23571113 \dots$}

Za začetek obravnavajmo število $0,23571113 \dots$, ki je zgrajeno iz zaporedja
blokov praštevil. Očitno decimalno število ni končno. Iz zapisa pa ne moremo razbrati ali 
je po nekem členu decimalno število periodično, zato ni očitno ali je število racionalno.
Trdimo naslednje:

\begin{trditev}
    \label{trditev praštevila}
    Število $0,23571113 \dots$ je iracionalno.
\end{trditev}

Za dokaz te trditve bomo potrebovali različico \textbf{Dirichletovega izreka}, ki pravi naslednje:

\begin{izrek}[Dirichletov izrek]
    \label{Dirichletov izrek}
    V vsakem zaporedju $ \lbrace a_n + b \rbrace_{n \in \N_0}$ naravnih števil, kjer sta $a$ in $b$
    tuji si naravni števili, je neskončno praštevil.
\end{izrek}

\noindent
{\em Dokaz trditve:\/} Naj bo $s \geq 0$ celo število. Po izreku \ref{Dirichletov izrek} vsako zaporedje
$ \{10^{s+1}k + 1\}$, $k \in \N$ vsebuje neskončno praštevil. Torej obstajajo praštevila
oblike $(k)\underbrace{00 \dots}_{s}1$, kjer števkam števila $k$ sledi $s$ ničel ter ena enica. 
Decimalno število $0,23571113\dots$ očitno vsebuje vse bloke take oblike za $\forall s \geq 0$. Z večanjem števila
$s$ narašča tudi število ničel v posameznem bloku, kar pomeni, da po še tako pozni decimalki zapis ne bo periodičen.
\qed

%% morda podamo še en dokaz???

%%%%%%%%%%%%%%%%%%%%%%%%%%%%%%%%%%%%%%%%%%%%%%%%%%%%%%%%%%%%%%%%%%%%%%%%%%%%%%

\section{Decimalna števila z naraščajočimi bloki}

% POPRAVI TA UVOD!!!
V prejšnjem razdelku smo dokazali, da je število $0,2371113 \dots$ iracionalno. Opazimo, da je to število
sestavljeno iz zaporedja blokov praštevil. Zanima nas kakšni morajo biti bloki iz katerega je sestavljeno decimalno
število, da bo iracionalno. 

Naj bo $1 \leq a_1 \le  a_2 \le \dots $ strogo naraščajoče zaporedje celih števil. 
Označimo: \[Dec\{a_k\} = 0,(a_1)(a_2)(a_3)... \  \ ;  a_k \in \Z \ k \in \N \]
Zanimajo nas lastnosti zaporedja $\{a_k\}$, ki nam zagotovijo, da bo $Dec\{a_k\}$ iracionalno.


\begin{izrek}\label{irac1}
    
    Če za strogo naraščajoče zaporedje celih števil $\{a_i\}_{i \in \N}$ velja 
    \[ \sum_{i=1}^{\infty} \frac{1}{a_i} = \infty ,\]
    potem je $Dec\{a_k\}$ iracionalno.
\end{izrek}

Ta izrek je posplošitev trditve \ref{trditev praštevila}, ki pravi, da je število $0,23571113 \dots$
iracionalno. Za zaporedje praštevil $2, 3, 5, 7, \dots$ namreč velja
\[
    \sum_{p \ \text{praštevilo}}\frac{1}{p} = \frac{1}{2} + \frac{1}{3} + \frac{1}{5} + \frac{1}{7} + \dots = \infty ,\]
kar je leta 1737 dokazal Leonhard Euler.

%% TU MORDA DODAJ SLIKICO

Za dokaz izreka \ref{irac1} bomo potrebovali naslednjo lemo.

\begin{lema}
    Naj bo $(b) = (b_1b_2b_3 \dots b_s)$ blok števil. Z $X = X(b_1b_2b_3 \dots b_s)$ označimo
    naraščajoče zaporedje pozitivnih celih števil, ki ne vsebujejo bloka števil (b). Potem 
    \[ \sum_{x \in X}^{\infty} \frac{1}{x}\] konvergira.
\end{lema}

\begin{opomba}
    \label{lema bloki}
    V lemi, v nasprotju z definicijo \ref{defincija blokov}, obravnavamo vsebovanost blokov v celih
    in ne decimalnih številih. Definicijo v ta namen prilagodimo tako, da le spremenimo potenco števila
    $10$ v vsoti. Celo število $x$ tako zapišemo kot 
    \[ 
        \begin{split}
            x & = \sum^n_{i=1} c_i 10^{n-i} = \\
            & = c_0c_1c_2 \cdots c_n \ .
        \end{split} \]
\end{opomba}

% MORDA DODAJ DEFINICIJO CELEGA DELA REALNEGA ŠTEVILA

\noindent
{\em Dokaz leme \ref{lema bloki}:\/} Označimo delno vsoto naše vrste 
\[ S_n = \frac{1}{x_1} + \frac{1}{x_2} + \dots \frac{1}{x_n} \ . \]
Naj bo $t \in \N $ tak, da velja $x_{t-1} \le 10^s \leq x_t$. Število števk bloka $(b) = (b_1b_2b_3 \dots b_s)$
je očitno $s$, število $10^s$ pa ima $s+1$ števk, kar pomeni, da imajo vsi $x_i$ za $i \geq t$
več števk kot je dolžina bloka. Delno vsoto sedaj preoblikujemo v
\[ S_n = \frac{1}{x_1} + \dots \frac{1}{x_t} + 10^{-s}(\frac{1}{\frac{x_{t + 1}}{10^s}} + \dots + \frac{1}{\frac{x_n}{10^s}})
    \le  \frac{1}{x_1} + \dots \frac{1}{x_t} + 10^{-s}(\frac{1}{ \lfloor\frac{x_{t + 1}}{10^s}\rfloor} + \dots + \frac{1}{\lfloor\frac{x_n}{10^s}\rfloor}) ,
\]
kjer smo v neenakosti upoštevali, da za poljuben $y \ge 0$ velja $y \ge \lfloor y \rfloor$.

Ker za vsak $\ t \le i \leq n$ velja $10^s \leq x_i$, bo $\lfloor \frac{x_i}{10^s} \rfloor$ pozitivno celo število.
To število lahko interpretiramo kot $x_i$ brez zadnjih $s$ števk. Ker vsi $x_i \in X$
ne vsebujejo bloka $(b)$, ga očitno ne vsebuje niti njegovih prvih nekaj števk. Od tod sledi, da 
$ \forall x_i, \ t \le i \leq n \ \exists x_y \in X$ tako da $ x_y = \lfloor \frac{x_i}{10^s} \rfloor$.

Novo nastala števila, pa so si lahko med seboj enaka. Tako bi na primer za $s=2$ veljalo 
$ \lfloor\frac{12345}{10^2}\rfloor = \lfloor \frac{12387}{10^2}\rfloor$.
Opazimo, da se blok $(b_1b_2b_3 \dots b_s)$ pojavi v vsaj eni od prvih $10^s$ naravnih števil.
Torej lahko poljuben $x_j \in X$ zadošča za največ $10^s - 1$ vrednosti $\lfloor \frac{x_i}{10^s} \rfloor$.

Po zgornjem premisleku lahko sedaj ocenimo izraz 
$(\frac{1}{ \lfloor\frac{x_{t + 1}}{10^s}\rfloor} + \dots + \frac{1}{\lfloor\frac{x_n}{10^s}\rfloor}) \le (10^s - 1)S_n$.
Če to oceno uporabimo na delni vsoti, dobimo:
\[
    \begin{split}
    S_n &\le \sum_{i=1}^t \frac{1}{x_i} + (10^s - 1)10^{-s}S_n \\
    S_n &\le 10^s \sum_{i=1}^t \frac{1}{x_i}
    \end{split}
    \]

%% BOLJŠE RAZLOŽI TALE SKLEPP!!!
Desna stran neenačbe je neodvisna od $n$, zato vrsta konvergira.
\qed
\\

\noindent
{\em Dokaz izreka \ref{irac1}:\/} Naj bo $\{a_i\}_{i\in \N}$ strogo naraščajoče zaporedje
celih števil za katero velja $\sum_{i=1}^{\infty}\frac{1}{a_i} = \infty$. Izrek bomo dokazali
s protislovjem. Predpostavimo, da je $Dec\{a_i\} = 0,(a_1)(a_2) \dots \in \Q$. To decimalno
število očitno ni končno, torej je periodično. 
To pomeni, da obstaja nek blok števil $(b) = (b_1b_2b_3 \dots b_s)$,
ki se v decimalnem zapisu periodično ponavlja od nekega mesta dalje. 

Definirajmo blok $(c)$ dolžine $2s$, tak da velja: če je $(b) = (11 \cdots 1)$, naj bo blok $(c)$
iz samih dvojic, v nasprotnem primeru pa naj bo $(c)$ iz samih enic.

Naj bo $Y = Y(c_1, c_2, \dots, c_{2s})$ zaporedje naravnih števil, ki ne vsebuje bloka $(c)$.
Razdelimo našo vsoto glede na vsbovanost v $Y$:
\[
    \sum_{i=1}^{\infty} \frac{1}{a_i} = \sum_{a \in Y}\frac{1}{a} + \sum_{a \notin Y} \frac{1}{a}
    \]

Po predpostavki vsota na levi strani enačaja divergira in po lemi vsota $\sum_{a \in Y}\frac{1}{a}$
konvergira. Od tod sledi, da bo $\sum_{a \notin Y} \frac{1}{a}$ divergirala.
To pomeni, da bo obstajalo neskončno $a_i$ v $Dec\{a_i\} = 0,(a_1)(a_2) \dots$, ki vsebujejo blok 
$(c_1c_2\dots c_{2s})$. Ta blok je dvakrat daljši kot blok $(b)$ in z različnimi števkami, zato se ne more
zgoditi, da bi bil $(c)$ vsebovan v $(b)$ ali sestavljal njegove dele. 
%% Tj. ne more se zgoditi, da bi na primer
Blok $(b)$ se zato ne bo ponavljal v neskončnosti, posledično $Dec\{a_i\}$ ne more biti periodično.
\qed

%%%%
\end{document}